\documentclass[12pt]{article}

\usepackage[margin=1in]{geometry}

\title{A computational analysis of vowel elision in Yoruba collocations}
\author{Hemant Gouni}

\begin{document}

\maketitle

% Implement GEN using bounded depth-first search??

In Yoruba, deletion occurs at morpheme boundaries. Specifically, when two
vowels become adjacent via concatenation of Yoruba morphemes, the first vowel
is generally deleted. However, [i] is always elided, regardless of whether it
is the first or second vowel in the sequence. Furthermore, however, when /u/
occurs at the end of the first morpheme it is always deleted, unless followed
by /i/ in which case /i/ is deleted. This is a well-known phenomenon to which
I provided an Optimality-Theoretic analysis in my previous paper.

I would like to extend this work in a further paper, not necessarily by
refining the theoretical basis of the analysis, but by offering a computational
interpretation of it. Specifically, I would like to create a program which,
when given a Yoruba underlying representation, creates the appropriate surface
form using the constraints and rankings derived in my analysis. This would be
an interesting project to explore because it may offer some insight into the
mechanical aspects of Optimality Theory which are not often addressed in the
literature. For instance, Optimality Theory relies on \textsc{Gen}, a mechanism
that produces all posssible surface forms from a given underlying one. These
forms are then evaluated in parallel against a given constraint ranking, with
the best ranked one being chosen as the winner. However, clearly, \textsc{Gen}
cannot truly produce all possible candidate output forms, because this would
take infinite time. Determining a heuristic that approximates \textsc{Gen}'s
functionality efficiently would be one such interesting aspect of this work.
The analysis from my previous paper would merely be used as a running example
to motivate and validate the results of the computational analysis, because I
am already familiar with it.

To summarize, the question I want to pursue is this: "What additional
considerations must be added to a standard Optimality-theoretic analysis in
order to make it amenable to automation?" I plan to start this work by
surveying any previous research on computer-aided phonological analysis,
particularly in the context of Optimality Theory. I will then attempt to write
my own computational analysis, perhaps using a language especially well-suited
to constraint-based programming like Prolog or Haskell, if I can find a way to
translate the necessary constructs (between my analysis and the language)
appropriately. If I succeed at doing this, I will then embark on the more
ambitious project of being able to encode abitrary constraints in the program.
This would allow it to generalize to Optimality-theoretic analyses beyond the
one I show here.

\end{document}
